\documentclass[conference]{IEEEtran}
\IEEEoverridecommandlockouts
\usepackage{cite}
\usepackage[utf8]{inputenc}
\usepackage[english]{babel}
 
\usepackage{hyperref}

\def\BibTeX{{\rm B\kern-.05em{\sc i\kern-.025em b}\kern-.08em
    T\kern-.1667em\lower.7ex\hbox{E}\kern-.125emX}}

\begin{document} 
    \title{
        Crime Prediction using Deep Learning\\[0.3cm]
        \large Project Proposal\\
        ELE494-09
    }

    \author{
        \IEEEauthorblockN{Yousif Khaireddin}
        \IEEEauthorblockA{b00063618\\}
        \and
        \IEEEauthorblockN{Nasir Khalid}
        \IEEEauthorblockA{b00065082\\}
    }

    \maketitle

    \section{Objective}

    The objective of our project is to develop a trained neural network
    that is capable of predicting where a future crime may occur and when it will occur within the city of Vancouver.
    \section{Datasets}

    For this project we will be narrowing it down to a specific city in which
    we will be predicting crimes. The decision of which city to pick comes down ultimately to
    the one with the largest and most accesible dataset.\\

    From our research we found a dataset on Kaggle that covers crimes in Vancouver from
    2003 to 2017. It contains 530,652 records in total and the columns are as follows:

    \begin{itemize}
        \item Type of crime
        \item Year
        \item Day
        \item Hour
        \item Minute
        \item Block of crime
        \item Neighbourhood of crime
        \item X Co-ordinate of crime in UTM Zone 10
        \item Y Co-ordinate of crime in UTM Zone 10
        \item Latitude
        \item Longitude\\
    \end{itemize}
    
    In addition to this data we also find another dataset about weather conditions in
    Vancouver.

    \begin{itemize}
        \item Humidity
        \item Temperature
        \item Windspeed
        \item Date and Time\\
    \end{itemize}

    We hope to use a combination of these two datasets to predict crime based on multiple variables. By linking the weather
    conditions of the second dataset along with the time, date and location of the first we can try to understand if weather plays a role
    in the occurance of crime and type of crimes being commited.\\

    \section{Network Inputs \& Outputs}

    Our plan is to create a neural network where the main inputs will be date, time and weather conditions. The network will then
    use all this data and return a probability of crime occuring. It will also return information on where the
    crime is most likely to happen within the city of Vancouver.

    We have not yet decided on what sort of network archiecture to use and we are keeping our options open. Some literature review
    is needed before arriving at a final decision.

    \section{Motivation}

    Such a neural network can help police and local municipality in predicting where crime occurs in the city and
    therefore they can prepare before hand on how to prepare police patrols and respond to a crime as quickly as possible. 

    \section{Conclusion}

    We believe that our project is capable of benefitting society as a whole as it can help reduce crime and help first
    responders. The datasets we found provide us with a multitude of variables to use during training and by the end we hope
    to have a successful deep learning network.
    
    \section{Links to Datasets}

    \href{https://www.kaggle.com/wosaku/crime-in-vancouver}{Vancouver Crime Dataset:}

    \begin{footnotesize}
        https://www.kaggle.com/wosaku/crime-in-vancouver\\    
    \end{footnotesize}
    

    \href{https://www.kaggle.com/selfishgene/historical-hourly-weather-data}{Vancouver Weather Dataset:}

    \begin{footnotesize}
        https://www.kaggle.com/selfishgene/historical-hourly-weather-data    
    \end{footnotesize}
    
\end{document}